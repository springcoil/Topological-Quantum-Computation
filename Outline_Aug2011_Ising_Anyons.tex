
\documentclass[preprint, 5p, authoryear, 12pt]{elsarticle}

\journal{Futures}
\usepackage{amsthm}
\usepackage{amsbsy}
\usepackage{amssymb}
\usepackage{amsfonts}
\usepackage{amsmath}
\usepackage{braket}
\usepackage{mathtools}
\usepackage[dvips]{lscape}
\usepackage{hyperref}% this enables jumping from a reference and table of content in the pdf file to its target
\usepackage{url}
\usepackage{graphicx}
\theoremstyle{plain}
 \begin{document}

\begin{frontmatter}
 \title{Outline of an article on Anyons and Ising models }

 
 \author[rvt]{Peadar Coyle\corref{cor1}}
 \ead{peadarcoyle@googlemail.com}
 
 \author[rvt]{Joost Slngerland}

 
 
 
 \cortext[cor1]{Corresponding author}


 
 
 
\begin{abstract}
This is the beginnings of an outline  
on Ising Anyons and the underlying Mathematics

\begin{keyword}
Topological Quantum Computation \sep Ising Anyons \sep TQFT \sep Category Theory \sep Fractional Quantum Hall Effect \sep Quantum Algebra
\end{keyword}


\end{abstract}


\end{frontmatter}

\section{Introduction}\label{sec1}
While in classical mechanics the exchange of two identical particles does not change the underlying state, quantum mechanics 
allows for more complicated behaviour. 2D quantum systems such as electrons confined between two semi-conductors or the superconductor based
model proposed in \cite{Blueprint_TQC_2010}, can give rise to exotic particle statistics, where the exchange of two identical (quasi) particles 
can in general be described by either Abelian or non-Abelian statistics. In the former, the exchange of two particles gives rise to a complex 
phase factor $e^{i\theta}$, where $\theta\;=\;0,\pi$
correspond to bosons and fermions respectively, and $\theta \;\neq\;0,\pi$
is referred to as the statistics of Abelian anyons.
\paragraph{} The statistics of non-Abelian anyons are described by $k\;\times\;k$ matrices acting on a \textit{degenerate sground state manifold}
with k $> \;1$ Since in general, two such unitary matrices A,B do not commute - these matrices form a non-Abelian group, hence the name.
\paragraph{} Anyons appear as emergent quasi-particles in fractional quantum Hall states and as excitations in microscopic models of frustrated
quantum magnets that harbor topological quantum liquids. 
It is proposed that for $\nu=\frac{5}{2}$ there will be Ising non-Abelian quasi particle statistics and $\nu=\frac{12}{5}$ there will be 
Fibonacci anyons. The proposals for topological quantum computation (TQC), the braiding of non-Abelain anyons are used to perform the unitary
transformations of a Q.C. See \cite{TQC_Review_2008} and references there in. 
 Detailed explanations of the Fibonacci Anyon models are provided in \cite{2009arXiv0902.3275T} and the further more complicated Ising model
is provided in \cite{Bomin2010PhRvL.105c0403B_IsingAnyons}. 
Let us consider the more complicated Tambora-Yamagomi Theory based upon the work of \cite{etingof-2002}
Expanation of 2D Quantum Systems
\begin{itemize}
\item Abelian Anyons
\item Nonabelian Anyons
\item Condensed Matter
\item Introduction to Quantum Computaton - include references
\end{itemize}

\subsection{Algebra}
This section is the sort of area my expertise may help the most with 
since I know a lot of Maths now.
\begin{itemize}
\item Tensor Category Theory
\item Fusion Rules
\item Fibonacci Anyons
\item Ising [similar to the $SU(2)_{k}$ that was mentioned
\item Magmas and Groups, some reference to the Mathematical Structures
\item Anyon models for finite groups
\end{itemize}
The Anyon Models for finite groups is ther real meat of this paper, since this 
model hasn't been completely milked. 
   \paragraph{} F Matrices and Pentagons/ R matrices  and Hexagons
Golden chain formulation
\begin{itemize}
\item T-L Algebras 
\item Jones Polynomials
\item Heisenberg Interactions
\end{itemize} 

\subsection{Numerical Analysis}
In the final section of the Fibonacci Intro, some numerical work and analytical work
analyzing the collective ground states formed by a set of Fibonacci anyons in the presence of the generalized Heisenberg interactions. 
\bibliographystyle{elsarticle-harv}

\bibliography{CoAlgebras,Analysis}

\end{document}
