\documentclass[a4paper,10pt]{article}
\usepackage[utf8x]{inputenc}
\usepackage{amsthm}
\usepackage{amsbsy}
\usepackage{amssymb}
\usepackage{amsfonts}
\usepackage{braket}
\usepackage{hyperref}
\usepackage{epsfig}
\usepackage{url}
\usepackage[dvips]{lscape}
\usepackage{natbib}
%opening
\title{Functional Analysis Prerequisites for PDE 1 and PDE 2}
\author{Peadar Coyle}
\theoremstyle{plain}
\newtheorem{theorem}{Theorem}[section]
\newtheorem{proposition}[theorem]{Proposition}
\newtheorem{lemma}[theorem]{Lemma}
\newtheorem{corollary}[theorem]{Corollary}
\newtheorem{remark}[theorem]{Remark}
\newtheorem{definition}[theorem]{Definition}
\newtheorem{notation}[theorem]{Notation}

                    
                


\begin{document}
\maketitle
\begin{abstract}
 
\end{abstract}
\section{Introduction}
I made this list in contact with my Professor. 
The purpose is to provide some sort of list, of the key and underlying Functional Analytic ideas for PDE,
which we can then use.
\begin{definition}[Metric Spaces]
A '''metric space''' is an \textbf{ordered pair} $(M,d)$ where $M$ is a \textbf{non-empty}
 set and $d$ is a \textbf{metric} on $M$, i.e., a function

$d : M \times M \rightarrow \mathbb{R}<$

such that for any $x, y, z\; \in \;M$, the following holds:
\begin{itemize}
\item $d(x,y) \ge 0$     ; (''non-negativity'') ,
\item $d(x,y) = 0$ \textbf{if and only if} $x = y$ 
\item $d(x,y) = d(y,x)$      (''symmetry'') and
 \item $d(x,z) \le d(x,y) + d(y,z)$ (''\textbf{triangle inequality}'') .
\end{itemize}
The first condition follows from the other three, since:
: $2d(x,y) = d(x,y) + d(y,x)\; \ge\; d(x,x) = 0.$

The function $d$is also called ''distance function'' or simply ''distance''. 
Often, $d$ is omitted and one just writes 
$M$ for a metric space if it is clear from the context what metric is used.
 
\end{definition}
\cite{RudinFA}
\begin{definition}
 
\end{definition}

\bibliography{PDE.bib}
\bibliographystyle{plain}
\end{document}
